    \chapter{Introduction}
    \addcontentsline{toc}{chapter}{Introduction}
    \setcounter{chapter}{1}
    %research goals
    %research motivation
    %thesis outline
    Colorectal cancer is one of the leading causes of cancer related deaths, causing approximately 900 thousand deaths worldwide per year (cite). Early detection thereof is as a consequence of significant importance. Polyps are often an early warning-sign of developing tumor, and early detection thereof can as a result significantly reduce fatality rates. Polyps are, however, often missed during colonoscopies, owing to the significant variability in the shapes an sizes of polyps, as well a the high degrees of similarity to surrounding tissue. Hence, automatic segment polyps via deep learning has the potential to significantly increase the likelihood of early detection and effective treatment. 
    
    However, clinical applications of deep learning are known to fail in deployment, despite exhibiting excellent performance during development. This is known as generalization failure, and is ubiquitous in the domain. While there has been a growing body of research dedicated to identifying and analyzing its root causes, there is still limited research into approaches to mitigating generalizability failure. 
    
    This thesis presents a novel approach to increasing generalizability, whereby the model is trained to not only minimize segmentation-loss, but also minimize the effects of the data being perturbed by an ensemble of transformations, including color-transformations, geometric transformations, additive noise, and adding extra polyps to the image using a GAN-inpainter. This endows the pipeline with the ability to more readily infer causally viable inductive biases by explicitly forcing the model to be robust to any combination of the aforementioned transformations. 
    
    Generalizability is then measured by evaluating several vanilla-pipelines consisting of several models on a number of separate datasets, which itsthen compared toroot causes, the results of the modified piresults show that (...)
    
   
%      What is the use of a Nifty Gadget?
        %Segmentation of polyps; importance
%      What is the problem?
        % Generalizability failure
%      How can it be solved?
        % Little research
%      What are the previous approaches?
        % ???
%      What is your approach?

%      Why do it this way?
%      What are your results?
%      Why is this better?
%      Is this a new approach?
%      Why haven't anyone done it before?