	The term "generaliZability" is used widely in literature, despite rarely being particularly well defined. Often, generalizability refers merely to the state of a predictor as "not overfitted". In other words, that it maintains sufficient performance across the training, validation, and test-sets. This, however, typically neglects the more salient aspects of the performance of the pipeline; namely how it behaves when deployed in practical settings, on data that may be distributed differently from the training dataset. Consider for instance the problem of detecting and classifying traffic signs. Though it is relatively trivial to achieve decent performance on such a task when training and evaluating on one specific dataset, it is another matter entirely to make sure the resulting predictors are robust to any and all forms of variability one might expect to see when deployed in a practical setting. If the training data was for instance collected from a region with a dry, temperate climate, it might not come as a surprise that it will not perform as well when deployed in an area prone to snowfall, fog, or generally low visibility. Of course, this could be mitigated by ensuring that the training data contains samples from a wide variety of climates, but this really only affects the robustness of the pipeline to differing climates. It does not necessarily ensure that the resulting predictors learn to ignore weather effects entirely, and as such it may nonetheless fail if it encounters something it has not been explicitly trained on. This is made especially evident in the study of black-box adversarial attacks: %examples ...
	        
	In other words, merely being robust to a limited class of perturbations or variability is not sufficient to deem a pipeline as generalizable. The pipeline must not merely learn to be right, but right for the right reasons. If this is achieved, robustness to distributional shifts follows. The system outlined above should in other words not only be robust to snow or rain or fog, but to be able to ignore the effects thereof entirely. A perfectly generalizable pipeline shouuld return a weather-invariant predictor every time, and for that matter maintain invariance to any and all non-destructive distributional shifts. 
	        
	The term generalizability will as such in this thesis refer to the ability to infer the right inductive biases from an incomplete dataset. This is as opposed to robustness, which denotes the ability of a pipeline to maintain its performance across certain distributional shifts. Generalizability is as a consequence not as much of a measurable quantity as much as it is an emergent property of a well-designed pipeline. 
	        
	This section will explore the concept of generalizability in further detail. It will outline how typical deep learning-based systems aim to achieve generalizability, why it nonetheless often fails to do so, and how one can analyse such generalizability failure.
	
			Naturally, however, real-world data is rarely neat enough for it to consistently abide by the iid assumption. Commonly encountered variation in real world data such as variable lighting conditions, class imbalance, image corruptions, noise, or other more subtle forms of distributional shift all result in structural misalignment of the training and deployment distributions (citation). Ideally, predictors should be robust to these sorts of changes, however evidently this is in not guaranteed by ERM (citation). ERM simply guarantees an iid-optimal predictor. While the difference is subtle, it is worth reemphasizing: empirical risk minimization only generalizes to data which is more or less identically distributed to the training data. Differently distributed or otherwise perturbed data, even that which is near imperceptible or at any rate inconsequentially different to the human eye, violates the iid assumption, and can as such not be expected to be classified correctly given a predictor trained via ERM. 
	
	To mitigate this, one could simply add more data to the pipeline through augmentation, or simply collecting more training data. This will lead to a better approximation of the true risk. This does not, however, solve the problem. The variability of the real world is not, unfortunately, easy to model merely through augmentations, and collecting sufficient data to cover every potential source of natural variability is infeasible, especially in medical domains. 
	%link this
	Consider for example a machine-learning pipeline wherein a model is trained to classify cows and camels. The dataset consists of cows, pictured in grass fields and pastures, and camels, pictured in deserts. To be generous, let us assume that we have sufficient quantities of data to ensure that the pipeline is perfectly invariant to the pose of the respective animals, to lighting conditions, geometric transforms, etc. One may then expect that the pipeline correctly learns to classify the two, and attains high accuracies, and indeed when evaluated on iid data, this would be entirely correct. However, what would then happen if one such predictor encountered a cow in the desert and a camel in a grass pasture? This constitutes a distributional shift, and as such we cannot expect reliable performance as detailed in \ref{erm}. Naturally, the predictor may have learned just fine exactly what constitutes a cow and a camel, but it might just as easily learned to associate deserts with camels and pastures with cows. And from a data perspective, both are equally correct interpretations. The immediate response to this may be to simply add some pictures with more varied backgrounds, but this once again would only serve to make the pipeline more robust to backgrounds. it would not guarantee that the pipeline learns the right inductive biases. The predictor may then for example instead learn that cows typically are black and white and camels usually beige, and then fail when it encounters a brown cow. One could keep adding more and more data, but there is not really any way of knowing when the pipeline is well enough specified by the data such that it starts returning predictors with the desired inductive biases. There are in simpler terms several "correct" interpretations of what separates the classes from a purely data-based perspective, each with their own inductive biases. There are as a consequence not just one risk-minimizing predictor, but a whole family of them. This is referred to as underspecification \cite{damour2020underspecification}.
	
	
		\subsubsection{Overfitting and underfitting}
	Overfitting and underfitting are perhaps the two most well-understood instances of generalisation failure. In simple terms, underfitting occurs when the span of the hypothesis space is insufficient to adequately model the relationships the model is intended to capture. Simple linear regression would, for instance, always fail to produce a generalisable predictotor of non-linear relationships between its input variables. Similarily, a neural network may be too shallow or too short, in which case the pipeline can never, even given infinite data and an optimal optimization process, find \(f\), since it simply does not exist in \(\mathcal{H}\) This is of course a fairly trivial problem to solve: simply increase the depth and/or width  of the model. And indeed, this is principally the reason why deep learning networks have proven to be superior to its more simple counterparts.
	

	Generalization failure is then in this case and by this line of reasoning entirely dependent on the structural misalignment and distributional shift corresponding to the change in imaging techniques as opposed to any erroneous logic in the pipeline itself. Ideally, the pipeline should of course detect patterns that generalize well regardless of lighting conditions, but it is not reasonable to expect the pipeline to draw this conclusion autonomously. Instead, the pipeline has to be "told" to keep this invariance in mind during training a priori. If some hypothetical change to the pipeline were to manage to induce this invariance, the structural misalignment between the dataset would no longer be an issue, ceteris paribus.

	The behavior that violations of assumptions \ref{underfit} and \ref{overfit} is well understood and fairly easy to detect, corresponding to underfitting and overfitting respectively, but violations of the remaining assumptions result in more subtle forms of generalization failure. 
		
	The general consensus is that generalization
	failure can in broad strokes be attributed to either underspecification or structural misalignment. The following sections will attempt to summarize and synthesize the analyses within the literature, and connect each of the generalization
	failure modes they identify to the above violations.

	In broad strokes, the generalisation failure modes identified in the literature can be categorized as belonging to one of the following phenomena:
		\begin{itemize}
			\item Pareidolia
			\item Underspecification
			\item Causal agnosticism 
		\end{itemize}
		
		, consider the problem of classifying images of cows and camels as the respective animals, wherein the dataset consists of images of cows in grass fields and pastures, and camels in deserts. To be generous, let us assume that we have sufficient quantities of data to ensure that the pipeline is perfectly invariant to the pose of the respective animals, to lighting conditions, geometric transforms, weather, etc. One may then expect that the pipeline correctly learns to classify the two, and attains high accuracies, and indeed when evaluated on iid data - i.e cows in fields and camels in deserts, this would be correct. However, what would then happen if one such predictor encountered a cow in the desert and a camel in a grass pasture? This, of course, constitutes a distributional shift, and as such we cannot expect the models to generalize. The predictor may have learned just fine exactly what constitutes a cow and a camel, but it might just as easily have learned to associate deserts with camels and pastures with cows. And from a purely statistical perspective, both are equally correct interpretations. From a causal perspective, however, it is of course entirely nonsensical to assume the respective animals are wholly defined by their surroundings.